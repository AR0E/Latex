\documentclass{antiquebook}


\usepackage{lipsum}
\usepackage{siunitx}
\usepackage[makeroom]{cancel}
\usepackage[math]{blindtext}

\author{Artur R. B. Boyago (aka Morcego)}
\title{{Moyses Vol. 2 - Soluções}}

\subfile{glossary.tex}

\begin{document}
	\frontmatter
	\maketitle

	\tableofcontents

	\mainmatter
	\pagestyle{fancy}

	\chapter{Estática dos Fluídos}

	\section{Q4}

	\chapter{Noções de Hidrodinâmica}

	\section{Q8}

	Queremos achar uma função $f(z)$ que descreva o formato do escoamento do fluído 
	duma torneira. Conceitualmente, o motivo do filete ter esse formato é que a água
	vai acelerando ao cair com aceleração $\mathbf{g}$, e por Bernoulli, isso implica 
	numa diminuição da seção transversal $A$ de forma que a massa total se retenha
	constante por unidade de tempo.

	\vspace{\baselineskip}

	\chapter{O Oscilador Harmônico}

	\section{Q1}

	\chapter{Oscilações Amortecidas e Forçadas}

	\section{Q1}

	\chapter{Ondas}

	\section{Q1}

	\section{Q5}

	Sabemos que a velocidade de fase $\mathbf{v}_\phi$ é $\sqrt{\mathbf{g}\lambda/2\pi}$.
	A velocidade de grupo é definida como:

	\begin{align*}
		\mathbf{v}_g = \partial_k \omega 
	\end{align*}
	
	Sabemos também que o número de onda é $k = 2\pi \lambda$, então na realidade temos:

	\begin{align*}
		\mathbf{v}_\phi = \sqrt{\frac{\mathbf{g} \lambda^2}{2\pi \lambda} } =
		\sqrt{\frac{\mathbf{g} \lambda^2}{k} } = \lambda \sqrt{\frac{\mathbf{g}}{k} }
	\end{align*}

	Mais, sabemos que a velocidade de fase e o número de onda se relacionam por
	$\mathbf{v}_\phi = \omega / k$, logo:

	\begin{align*}
		\mathbf{v}_\phi = \frac{\omega}{k} = \lambda \sqrt{\frac{\mathbf{g}}{k} }
	\end{align*}

	Logo, podemos descobrir $\omega$ para depois diferenciarmos em respeito a $k$:

	\begin{align*}
		\omega = \lambda k \sqrt{\frac{\mathbf{g}}{k} } = \lambda \sqrt{k\mathbf{g} }
	\end{align*}

	E sua derivada em respeito a $k$:

	\begin{align*}
		\partial_k\omega &= [\lambda(k\mathbf{g})^{1/2}]' \\
		&= \lambda (1/2)(k\mathbf{g})^{-1/2} \\
		&= \frac{\lambda}{2\sqrt{k\mathbf{g}}} 
	\end{align*}

	O que completa a prova.
	
	\chapter{Temperatura}

	\chapter{Calor. Primeira Lei da Termodinâmica}

	\section{Q1}

	No topo da catarata se tem apenas a energia potencial gravitacional
	$\mathbf{E}_p$, um leve fluxo da água, e uma pequena quantidade
	de temperatura. Desprezando esses dois, podemos considerar


	\chapter{Propriedade dos Gases}

	\chapter{Noções de Mecânica Estatística}

	\section{Q1}
	

\end{document}