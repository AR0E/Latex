\documentclass{antiquebook}


\usepackage{lipsum}
\usepackage{siunitx}
\usepackage[makeroom]{cancel}
\usepackage[math]{blindtext}

\author{Artur R. B. Boyago (aka Morcego)}
\title{{Moyses Vol. 1 - Soluções}}

\subfile{glossary.tex}

\begin{document}
	\frontmatter
	\maketitle

	\tableofcontents

	\mainmatter
	\pagestyle{fancy}

	\chapter{Movimento Unidimensional}

	\section{Q1}
	
	O problema estabelece que a tartaruga anda continuamente até o final, mas que a lebre tem uma 
	pequena parcela de tempo de descanso. Portanto, o tempo que demora a tartaruga chegar ao final deve
	ser menor ou igual ao tempo da lebre chegar ao final para que ela ganhe a corrida. Portanto, vamos achar o tempo de corrida da tartaruga, e igualar ao tempo de corrida da lebre:

\begin{align*}
	&(\text{Tempo de corrida da tartaruga}) = \frac{\text{Dist. total}}{v_t} \\
	&(\text{Tempo de corrida da lebre}) = \frac{\text{Dist. ao descanso}}{v_l}+ (\text{Tempo de descanso} \ t_s)+\frac{(\text{Dist. total - Dist. ao descanso})}{v_l}
\end{align*}	

	Igualando os dois, podemos achar uma expressão pro tempo de descanso $t_s$ e substituir os valores:

\begin{align*}
	&\frac{D}{v_t} = \frac{d}{v_l} + t_s + \frac{D-d}{v_l} \\
	&\frac{D}{v_t} - \frac{D}{v_l} = t_s \\
	&(600 \ \si{m}) \left ( \frac{1}{1,5 \ \si{m/min}} - \frac{1}{30 \ \si{km/h}} \right ) \approx 6\si{h}38\si{min}
\end{align*}

	\section{Q2}

	Queremos saber o tempo para atingir uma velocidade sob aceleração constante, 
	que pode ser achado por $v(t)=at$, e
	depois a distância total, que é $at^2/2$.

	\begin{align*}
		a = \frac{100 \ \si{km/h}}{4 \ \si{s}} \approx \frac{27,7 \ \si{m/s}}{4 \ \si{s}} \approx 6,92 \ \si{m/s^2} 
	\end{align*}

	Esse $a$ é $\approx 70\%$ da gravidade média da terra. O tempo é
	então:

	\begin{align*}
		(100 \ \si{km/h}) = (6,92 \ \si{m/s^2})t \rightarrow t \approx 4 \si{s}
	\end{align*}

	Substituindo para achar a distância:

	\begin{align*}
		x(t) = \frac{1}{2}(6,92 \ \si{m/s^2})(4\si{s})^2 \approx 2,41 \ \si{km}
	\end{align*}


	\section{Q3}



	\section{Q4}

	\section{Q5}

	\section{Q6}

	\chapter{Movimento bidimensional}

	\section{Q1}

	\section{Q2}

	\section{Q3}

	Queremos provar a desigualdade:

	\begin{align*}
		| |\mathbf{a}| - |\mathbf{b}|| \le
		| \mathbf{a}+\mathbf{b} | \le
		| \mathbf{a} | + | \mathbf{b} |
	\end{align*}

	Essa identidade é trivial por lógica; o primeiro termo sempre vai ter dois números
	estritamente positivos se subtraindo e gerando outro positivo, no segundo temos dois números
	formando um estrito positivo, e no terceiro \emph{sempre} temos dois números estritamente
	positivos se somando.

	\vspace{\baselineskip}

	Usando algebra geométrica, sabemos que
	$|\mathbf{a}| = \mathbf{a}^2$, logo, simplesmente
	fazendo quadrados de tudo:

	\begin{align*}
		(\mathbf{a}^2-\mathbf{b}^2)^2 \le (\mathbf{a}+\mathbf{b})^2 \le
		\mathbf{a}^2 + \mathbf{b}^2
	\end{align*}

	\chapter{Os Principios da Dinâmica}

	\section{Q1}

	Isso é só uma prova das \emph{lei dos senos}. Sabendo que $\mathbf{F}_1+\mathbf{F}_2+\mathbf{F}_3=0$,
	podemos usar algebra geométrica rapidamente; tome o produto externo $\wedge$ em tres partições
	dessa equação:

	\begin{align*}
		&(\mathbf{F}_1+\mathbf{F}_2+\mathbf{F}_3)\wedge\mathbf{F}_1=0\wedge \mathbf{F}_1 \\
		&(\mathbf{F}_1+\mathbf{F}_2+\mathbf{F}_3)\wedge\mathbf{F}_2=0\wedge \mathbf{F}_2 \\
		&(\mathbf{F}_1+\mathbf{F}_2+\mathbf{F}_3)\wedge\mathbf{F}_3=0\wedge \mathbf{F}_3 
	\end{align*}

	Termos da forma $\mathbf{x} \wedge \mathbf{x}$ são $0$, logo temos:

	\begin{align*}
		&\mathbf{F}_2\wedge\mathbf{F}_1+\mathbf{F}_3\wedge\mathbf{F}_1 =0 \\
		&\mathbf{F}_1\wedge\mathbf{F}_2+\mathbf{F}_3\wedge\mathbf{F}_2 =0  \\
		&\mathbf{F}_1\wedge\mathbf{F}_3+\mathbf{F}_2\wedge\mathbf{F}_3 =0  
	\end{align*}

	Se colocarmos os segundos termos no lado do zero, e depois tomarmos suas magnitudes, temos:

	\begin{align*}
		&|\mathbf{F}_2\wedge\mathbf{F}_1| = |\mathbf{F}_3\wedge\mathbf{F}_1|  \\
		&|\mathbf{F}_1\wedge\mathbf{F}_2| = |\mathbf{F}_3\wedge\mathbf{F}_2|  \\
		&|\mathbf{F}_1\wedge\mathbf{F}_3| = |\mathbf{F}_2\wedge\mathbf{F}_3| 
	\end{align*}

	Pela regra que $|\mathbf{a}\wedge\mathbf{b}| = |\mathbf{a}| |\mathbf{b}| \cos \theta_{ab}$,
	temos então a conclusão final da lei dos senos:

	\begin{align*}
		\frac{\sin \theta_{23}}{\mathbf{F}_1} = \frac{\sin \theta_{31}}{\mathbf{F}_2} = \frac{\sin \theta_{12}}{\mathbf{F}_3}
	\end{align*}

	\section{Q8}

	O martelo originalmente tem uma energia cinética de $m\mathbf{v}^2/2$, e pelo
	trabalho $W$ que a madeira faz no prego, essa energia é totalmente dissipada
	pra zero. O trabalho é a força média $\mathbf{F}_m$ vezes o deslocamento
	$l$. Queremos achar a proporção:

	\begin{align*}
		\frac{(\text{For. med.})}{(\text{Peso do martelo})} = 
		\frac{\mathbf{F}_m}{{m}\mathbf{g}} = \frac{W/l}{{m}\mathbf{g}}
	\end{align*}

	Sabendo que:

	\begin{align*}
		\frac{1}{2}m\mathbf{v}^2 - \mathbf{F}_m l = 0 \rightarrow
		\mathbf{F}_m = \frac{1}{2l}m\mathbf{v}^2
	\end{align*}

	Portanto podemos substituir:

	\begin{align*}
		\frac{ \left ( \frac{1}{2l}m\mathbf{v}^2  \right ) }{m\mathbf{g}}
	\end{align*}

	Para acharmos $\mathbf{v}$, o problema disse que é a velocidade
	final do martelo depois duma queda de altura $h$. Essa velocidade é
	$\sqrt{2\mathbf{g}h}$, logo:

	\begin{align*}
		\frac{ \left ( \frac{1}{2l} \cancel{m} \mathbf{v}^2  \right ) }{\cancel{m}\mathbf{g}} = \frac{\mathbf{v}^2}{2l\mathbf{g}} = 
		\frac{2\cancel{\mathbf{g}}h}{2\cancel{\mathbf{g}}l} = \frac{h}{l}
	\end{align*}



	\chapter{Rotações e Momento Angular}

	\section{Q1}

	Esse problema é relativamente conceitual então serei breve.
	O vetor $\mathbf{c}=\mathbf{a} \times \mathbf{b}$ é um \emph{vetor axial} que representa
	o \emph{dual pseudoescalar} do bivetor $\mathbf{a} \wedge \mathbf{b}$. Por isso
	$|\mathbf{c}| = |\mathbf{a}\wedge \mathbf{b}|$, ou seja, seu comprimento é proporcional
	a área do bivetor, e pode ser então descrito como $i(\mathbf{a}\wedge\mathbf{b})$.

	\vspace{\baselineskip}

	Sobre o segundo problema, se você somar vários bivetores em direções opostas com a mesma
	magnitude, evidentemente você irá retornar zero. Uma nota, irei usar bivetores ao invés
	de vetores axiais em todo o resto do livro por que os acho intoleravéis.

	\section{Q2}

	O \emph{\textbf{momento de dipolo}} é $\mathbf{p} = q\mathbf{d}$, ou seja, apenas uma 
	diferença de posições com escala por carga. Sabemos que o torque é o bivetor 
	$\mathbf{\tau} = \mathbf{f} \wedge \mathbf{x}$, e sabemos que a força $\mathbf{f}$
	do campo elétrico $\mathbf{E}$ é simplesmente $q\mathbf{E}$, logo o torque:

	\begin{align*}
		\mathbf{\tau} &= (q\mathbf{E}) \wedge \mathbf{d} \\
		&= \mathbf{E} \wedge (q\mathbf{d}) \\
		&= \mathbf{E} \wedge \mathbf{p}
	\end{align*}
	
	Sobre a energia potencial $U$, sabemos que ela deve ser:

	\begin{align*}
		U \propto \int_\gamma \mathbf{F}\cdot d\mathbf{x}
	\end{align*}

	Mas queremos uma relação ao torque $\mathbf{\tau}$.

\end{document}