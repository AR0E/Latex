\chapter{Introduction}

\paragraph{The current market.}
The (digital) world is \textit{covered} in meshes: evaluated at more than
USD $238$\textit{ billion} \cite{360iMarket2025ComputerGraphics} in
2024, and expected to reach around USD $372$ \textit{billion} by 2030,
the computer graphics community is in virtually every
segment of the industrial and scientific world. From 
\textit{Finite Element Methods} (FEM) employed for the
simulations of fluids,
stellar coronae and static deformations, to
stunning award-winning visual effects accomplished
by \textit{Weta} and \textit{Pixar}, and a variety of robotics,
machine learning and medical applications, \textit{\textbf{meshes}}
are the most used backbone to all these areas, underlying
the majority of the used software and mathematical
constructions.

\spa

The applications of most interest
to this study relate to \textit{2-triangulated} (triangle) meshes used in practical
geometrical steps to many algorithms and artistic endeavours, be it
texture wrapping, variational cutting, and mesh reconstruction
from point primitives, as well to the disciplines of physical simulations 
with mesh based methods, such as \textit{FEM/BEM/ISPEM}, which involve
\textit{3-triangulated} (tetrahedron) as well as standard 2-triangulated meshes.
A sample application is the one in producing
topologically stable \textit{geodesic tracings} on broken or highly deformed
objects, with the intent of later producing accurate transformations
or calculating partial differential equations on them by means of a metric
compensation, and the calculation of fluid flow on curved surfaces:

\begin{figure}[h]
    \centering
    % First image
    \begin{subfigure}[t]{0.4\textwidth}
        \centering
        \includegraphics[width=\linewidth]{images/bunny_heat.png}
    \end{subfigure}
    % Second image
    \begin{subfigure}[t]{0.4\textwidth}
        \centering
        \includegraphics[width=\linewidth]{images/soap_film.png}
    \end{subfigure}
    
    \caption{Production of geodesics on a bunny mesh by means of the
    \textit{heat method} \cite{Crane:2017Heat}, and simulation of a soap
    film on Costa's minimal surface \cite{chern_cohomology}.
    Source: Crane, Chern.}
    \label{fig:two-images}
\end{figure}


Given their plethora of applications and usefulness, it makes sense to develop
very robust mathematical methods to manipulate meshes in physical
and mathematical contexts. However, the difficulties for this rise rapidly
when the context for current industrial standards come into play:
with over \textit{a 100 different file formats} (\texttt{.obj, .wav} ...) \cite{wiki_Files},
and more than \textit{30 major} software packages of manipulating 
said files, we then have 4 different coordinate system conventions,
different spline parametrization conventions, and dozens of small
variations in geometric primitives and data structures.

\spa

\paragraph{Mesh quality.} Further, supposing you take all this into account, we still have
the actual \textit{quality} of the meshes to take into consideration.
The vast majority of geometric data, period, is extremely \textit{irregular},
being mostly harvested from physical sensors and observation equipment, but
even purely digitally produced data is full of degenerate artifacts due
to poor practices and mesh production algorithms.
This means a large part of geometry processing algorithms on meshes
\textit{fail} for big datasets, since they can't handle all
the possible edge cases unless otherwise explictly told to.


\begin{figure}[h]
    \centering
    % First image
    \begin{subfigure}[t]{0.4\textwidth}
        \centering
        \includegraphics[width=\linewidth]{images/mesh1.png}
    \end{subfigure}
    % Second image
    \begin{subfigure}[t]{0.46\textwidth}
        \centering
        \includegraphics[width=\linewidth]{images/mesh2.png}
    \end{subfigure}
    
    \caption{Difficulties in mesh quality vary from extremely high
    polygon density, sheared triangles, unpatched holes, lack of
    connectivity, highly skewed geometry, etc.
    Source: Nicholas Sharp.}
    \label{fig:two-images}
\end{figure}

Therefore, either you need to \textit{manually} clean up your meshes,
or produce one from scratch given some other geometric data, which is
a very laborious, time consuming procedure, requiring artistics and technical
expertise. Consider the manual meshing process of a \textit{Formula 1 car}
\cite{beta-cae-cfd-formula-car}, which even for simple models
can take up to a week, or almost any mesh production for CFD simulation.

\spa 

A great part of mesh processing algorithms still need to be
manually tuned and selected by experts, be it researchers or artists,
to be used, and this manual tweaking can often \textit{break}
the desired properties of the algorithm in the first place.
The greatest offenders are the so-called \textit{variational
crimes} \cite{szabo2024-not-all-models_crimes} in many FEM solvers, which break the behavior and accuracy
of variational (e.g Garlekin) methods by poor geometric
considerations, all of which can be fixed by more geometric-centry
algorithms, such as by \textit{Finite Element Exterior Calculus} \cite{Holst_2012_variationalCrimes}; these errors are \textit{exceptionally}
dangerous, because, as reported many times \cite{suri2024-cracks-sciam_variationalCrimes},
they can introduce very dangerous vulnerabilities in airplanes, boats, and
other critical equipment and infrastructure, since the related models
don't actually converge to the physically real answer, leading to
underestimations of stress, and etc. Therefore, even more than just
convencience, new geometric methods are needed for safety.


\spa

A simpe illustration of this phenomena is by the distinction in the simulation
of \textit{hyperbolic} or \textit{elliptic} differential operators.
Altough elliptic operators are generally as smooth as the input data and
conditions permit them to be, hyperbolic operators usually contain
naturally discontinuous and badly-behaved solutions \cite{singular_FEM}, 
such as shocks
and solitons, therefore many methods fail to even properly
capture the actual node-behvaior, much less convergence, of the solutions.
An example is in the solutions of the \textit{Poisson problem},
a smooth "screened" Laplacian problem,
and of the \textit{Inviscid Burguers problem}, a fairly simple
nonlinear wave equation with dissipation in fluid dynamics.
Because of the presence of modes proportional to the
gradient, it develops sharp gradients which then become
\textit{shocks} \cite{Cameron2025burgers}.

\begin{figure}[h]
    \centering
    \includegraphics[width=1.0\textwidth]{images/hyperbolic_operators.png}
    \caption{Comparison of an elliptic operator, the Poisson problem,
    and the hyperbolic inviscid Burguers problem, which contains
    jump discontinuities and shocks Source: Author}
    \label{fig:two-vertical-images}
\end{figure}


\section{Justification}


\paragraph{Mesh agnosticism}
All of this leads to the question of how to efficiently produce new
algorithms that can process \textit{ any} mesh, regardless
of its quality or origin? It is of note how appreciably hard this
problem is, given that in such an active landscape it hasn't been
satisfactorily solved yet, but it is not \textit{impossible}.
In the last 10-15 years, new techniques have been slowly implemented
by research groups such as the \href{https://github.com/GeometryCollective}{\textit{Geometry Collective}} and \href{https://scholar.google.com/citations?hl=de&user=FNe48c8AAAAJ&view_op=list_works&sortby=pubdate}{\textit{Ulrich Pinkall's group}}
and many more as to create \textit{mesh agnostic} methodologies.

\spa

These are done
under the general guise of \textit{discrete analysis}
and \textit{discrete differential geometry}, or \textit{DDG}. 
The \textit{lemma} of DDG is similar to the "quantization" procedure of physics; 
given the \textit{discretization} of a smooth shape, is it possible to,
in the limit of \textit{homogeneity}, that is, of \textit{infinite} mesh resolution,
recover all the given results of regular geometry
\cite{Crane_discrete}? Further, is it possible
to make similar analogues to theorems even in low resolution scenarios?


\begin{figure}[h]
    \centering
    \includegraphics[width=0.8\textwidth]{images/Lema_DDG2.png}\\[1.7ex] % 1ex adds vertical spacing
    \includegraphics[width=0.7\textwidth]{images/discrete_curvature.png}
    \caption{The \textit{lemma} of DDG (top), and the variety of possible
    curvature measures on a discrete polygonal curve (bottom). Source: Author, Crane.}
    \label{fig:two-vertical-images}
\end{figure}

For example, naturally, the \textit{extrinsic curvature}
of an object may defined on a parametrized curve
$\gamma(s)$ as $\kappa = \langle N, \Ddot{\gamma}(s)\rangle$, that is, the normal projected
onto the "acceleration". For any
$C^2-$curve, this is of course well-defined
and well behaved, but what of a discrete,
polygonal curve? Any attempt to take the second
derivative of a piece-wise discrete function
will lead to the curvature being \textit{infinite}.
Nevertheless, we can produce \textit{multiple}
"curvature measures of said" curve that,
in the infinite polygon limit, mimic the
exact smooth curvature \cite{Crane:2013:DGP}.

\spa

Taking this idea to its limit, we then
finally arrive at the project's full proposal,
which is a subproject of the area of DDG
to a full coverage of \textbf{discrete mesh differential operators}, which is a
map between the often known differential
operators in the smooth $C^\infty(\mathcal{M})$
setting to the discrete, piecewise setting.
This mapping allows for a full reproduction
of the results of the theory of \textit{partial
differential equations, algebraic topology} and more
onto the computational setting.


\spa

\paragraph{Spectral geometry processing.} Let's use a simple example of the discretization procedure,
and perhaps its most important.
Given the regular \textit{Hodge-Laplace} operator ($\bf{\Delta}$) on a smooth
manifold $\mathcal{M}$, we may define the \textit{discrete} equivalent
by means of a \textit{intrinsic Delaunay triangulation} procedure \cite{laplacian0}. 

\begin{definition}[Discrete Hodge-Laplace operator on 0-forms]
Let $(\mathcal{M},g)$ be a smooth oriented Riemannian surface and 
$\Delta := \delta d + d \delta$ the Hodge--Laplace operator, where
$d$ is the exterior derivative and $\delta = (-1)^{nk+1} \star d \star$ is the codifferential
on $k$-forms in dimension $n$. 
On $0$-forms $f \in \Omega^0(\mathcal{M})$, this reduces to
\[
   \Delta f = \delta (df),
\]
the usual Laplace--Beltrami operator on functions.

\spa

Let $\bf M=(V,E,F)$ be a simplicial surface mesh approximating $\mathcal{M}$, with
vertices $\mathbf{V}$, edges $\mathbf{E}$, and triangular faces $\mathbf{E}$. For an interior edge 
$ij \in \mathbf{E}$ shared by two triangles $(i,j,k)$ and $(i,j,\ell)$, let
$\alpha_{ij}, \beta_{ij}$ denote the angles opposite $ij$ in those triangles. 
The \emph{cotangent weight} for edge $ij$ is
\[
   w_{ij} \;=\; \tfrac{1}{2}\big(\cot \alpha_{ij} + \cot \beta_{ij}\big),
\]
while for a boundary edge $ij$ (incident to a single face with opposite angle
$\alpha_{ij}$) one sets $w_{ij} = \tfrac{1}{2}\cot \alpha_{ij}$.

\spa

The \textbf{\emph{discrete Hodge--Laplace operator}} (a.k.a.\ cotangent Laplacian)
acting on $0$-forms $u:V\to\mathbb{R}$ is the matrix
$L \in \mathbb{R}^{|V|\times |V|}$ with entries
\[
   (L f)_{ij} = \frac{1}{2a_i} \sum_{i,j \ \in \ \mathbf{E}} (\cot \alpha_{ij} +
   \cot \beta_{ij}) \ (f_i -f_j)
\]
Then for any function $u:V\to \mathbb{R}$, the discrete Laplacian is 
$(Lu)(i) = \sum_{j:\, ij\in E} w_{ij}\,\big(u(i)-u(j)\big)$.
This operator is symmetric, has rows summing to zero, and
converges to the smooth Laplace--Beltrami operator under mesh refinement.
\end{definition}

\spa

\begin{figure}[h]
    \centering
\includegraphics[width=1.1\textwidth]{images/cotagent_laplace.png}
    \caption{Comparison of the solutions of the discrete
    cotagent laplacian on a discretized mesh and the smooth
    laplacian. Source: Author}
    \label{fig:two-vertical-images}
\end{figure}

This operator is very much the "swiss-army knife"
\cite{laplacian2} of geometry processing algorithms; the
main motivation comes to that the Laplacian is the lowest
order differential operator that is self-adjoint and elliptic,
and has the largest significance physically as one can
prove that any operator that commutes with rotations
and translations is just a linear combination of the
laplacian. For the general eigenvalue Helmholtz problem
where a solution will always be $u_k(\mathbf{x},t) = \alpha_k(t) \phi_k(\mathbf{x})$:

\begin{equation*}
    \mathbf{\Delta} \phi_k = \lambda_k \phi_k, \quad \lambda_k \in \mathbb{C}
\end{equation*}

We may have solutions \cite{canzani2013analysis_laplacian}:

\begin{equation*}
    \alpha_k(t) = \begin{cases}
        \exp(-\lambda_k t) \quad            &\text{Heat equation}\\
        \exp(i \sqrt{\lambda_k} t) \quad    &\text{Wave equation}\\
        \exp(i \lambda_k t) \quad           &\text{Schrödinger equation}
    \end{cases}
\end{equation*}

This permits a variety of deductions, from estimating the boundary
behavior (in his famous essay \textit{Can One Hear the Shape of a Drum?} \cite{Kac1966drum})
as well as estimating the actual deformation properties
of the volume and area elements, estimating distances,
so on and so forth.

\spa

However, in the same vein as the discrete extrinsic curvature,
there are \textit{many} Laplacian choices that need to respect
the smooth laplacian, all of which with their own pros and cons
\cite{laplacian1} (The positive-semi definitess was only recently
fixed with the intrinsic Delaunay Laplacian, for example). That is,
there are many equivalent choices that approach the limit of the smooth
one. 

\spa

This introduces the problem of choice: what are the \textbf{error
metrics} one needs in order to determine the best discrete operator?
More, what is a general procedure of constructing them if they're
generally non-unique? This lack of well-posedness forces us
to make some kind of general \textit{availability} criteria,
as to wheter or not the properties of an operator $D$ in the smooth
setting are availble to us in the discrete one.

\paragraph{Beyond $0-$forms.}
Beyond the simple Laplacian, we also have the
famous \textit{Dirac operator}, its famous
"square root", and the fundamental
equation of relativistic quantum theory.

Folloiw

\begin{definition}[Discrete extrinsic Dirac operator]



\begin{equation}
    D \phi \def -\frac{dg \wedge df}{}
\end{equation}

We $dz \wedge d\bar{z} = -2i|df|^2$

\end{definition}


















\section{Previous work}

\paragraph{Discrete analysis.}
A large amount of work on discrete operator theory has already been
done, even previous to the advent of much of modern computing, by means
of \textit{discrete analysis}. Examples include the famous
\textit{Regge Calculus} of general relativity \cite{cuzinatto2019introductionreggecalculusgravitation},
or in the study of topology and homotopical algebra by means
of simplicial complexes and sheaves \cite{castel2021_morse}. The
\textit{expansion} of these discrete theories for computational
purposes has also been massively sucessful and applied
in a dozen or so different mathematical areas; a lot of these
results however are quite \textit{scattered}, so a thorough
literature review is necessary.

\spa

Specific to our purposes is most of the literature related to
the \textit{discrete exterior calculus} (DEC) 
\cite{Wang:2023:ECIG, Pt_kov__2021_exterior}, including
extensions for bundle-valued elements \cite{braune2025discreteexteriorcalculusbundlevalued}
and methods 
therein as to implement all of the basic differential operators and
simplicial complex theory in computers, such as
\textit{discrete connections} \cite{Crane:2010:TCD},
\textit{homology} \cite{homology_discrete, dean_homology} and \textit{cohomology} measures,
and other differential geometric quantities \cite{algebraic_topology}.
A couple of works have already implement partly or purely functional libraries
of DEC such as in the Rust language \cite{wirth2025rustimplementationfiniteelement},
or more sophisticated Clifford algebraic libraries such as in Javascript
\href{https://github.com/enkimute/ganja.js}{ganja.js}.

\spa

\paragraph{Differential taxonomy.}
Somewhat general frameworks for discrete differential operators
in meshes are already present
\cite{discrete_operators, discrete_operator2, discrete_operators3}, including well posed
parabolic differential operators \cite{Parabolic_Discrete}.
Of particular interest to the the author is in the distinctions
and manipulation of \textit{intrinsic, extrinsic, scalar} and \textit{hypercomplex}
operators. The first two categories verify how much on the normal vectors of the
given mesh the operator depends on; the more it is solely defined
by edge/vertex relationns, the more intrinsic it is, and the two
other categories distinguish between operators that act on and
output
$0-$forms $f(\mathbf{x})$, regular functions, and operators that take as either
input or output $(p,q)-$valued forms or general hypercomplex numbers, such
as \textit{quaternions} \cite{Crane:2011:STD, conformal_chern, palmer2021framefieldoperators}, spinors
\cite{Liu:2017:DOE, spinorsMesh}, and multivectors/Clifford
numbers in general \cite{discrete_clifford, budinich2018clifford, discrete_clifford2,360iMarket2025ComputerGraphics, Hypercomplex_discrete}.
Of future interest is the implementation of techniques of the physical
sciences \cite{krasilshchik1998homologicalmethodsequationsmathematical}
into these methodologies.

\spa

Based on previous surveys \cite{WANG201941}, we have made a small table
of the current landscape of operators (not ehxaustive):

\begin{table}[h]
\centering
\small
\setlength{\tabcolsep}{8pt}
\renewcommand{\arraystretch}{1.15}

% left column: fixed width & vertically centered; others: auto-wrap
\begin{tabularx}{\linewidth}{M{3.1cm} Y Y}
\toprule
& \multicolumn{1}{c}{\textbf{Intrinsic}} & \multicolumn{1}{c}{\textbf{Extrinsic}} \\
\midrule
\textbf{Scalar} &
\diagcell{%
\textbullet\enspace Laplace--Beltrami (cotan)\\
\textbullet\enspace Graph Laplacian\\
\textbullet\enspace Anisotropic Laplacians%
}{%
\textbullet\enspace Hessian operators\\
\textbullet\enspace Schr\"odinger/Hamiltonian\\
\textbullet\enspace Connection Laplacian%
}
&
\twocol{%
\textbullet\enspace Dirichlet-to-Neumann\\
\textbullet\enspace Volumetric Laplacian\\
\textbullet\enspace Shape operator%
}{%
\textbullet\enspace Mean curvature flow\\
\textbullet\enspace Curvature Laplacian%
}
\\
\midrule
\textbf{Hypercomplex} &
\twocol{%
\textbullet\enspace Intrinsic Dirac\\
\textbullet\enspace Quaternionic Laplacian%
}{%
\textbullet\enspace Spinor Dirac\\
\textbullet\enspace Cauchy--Riemann / Beltrami%
}
&
\diagcell{%
\textbullet\enspace Extrinsic Dirac\\
\textbullet\enspace Relative Dirac%
}{%
\textbullet\enspace Conformal deformation ops\\
\textbullet\enspace Spinorial shape ops%
}
\\
\bottomrule
\end{tabularx}

\caption{Dictionary of mesh operators by intrinsic/extrinsic and scalar vs.\ hypercomplex.}
\label{tab:mesh-operators}
\end{table}

\spa 

\begin{figure}[h]
    \centering
\includegraphics[width=1.0\linewidth]{images/dirac_operator_shapes.png}
    \caption{Gradient between intrinsic scalar (Laplace) and
    extrinsic hypercomplex (Dirac) operators. Source: Crane.}
    \label{fig:two-vertical-images}
\end{figure}
