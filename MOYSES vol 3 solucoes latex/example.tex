\documentclass{antiquebook}


\usepackage{lipsum}
\usepackage{siunitx}
\usepackage{physics}
\usepackage[makeroom]{cancel}
\usepackage[math]{blindtext}

\author{Artur R. B. Boyago (aka Morcego)}
\title{{Moyses Vol. 3 - Soluções}}

\subfile{glossary.tex}

\begin{document}
	\frontmatter
	\maketitle

	\tableofcontents

	\mainmatter
	\pagestyle{fancy}

	\chapter{A Lei de Coulomb}

	\section{Q1}

	A razão entre as atrações mútuas de um elétron e próton, gravitacionalmente e eletrostaticamente,
	são as razões entre as magnitudes das duas forças:

	\begin{align*}
		\frac{|(\text{F. eletro.})|}{|\text{(F. grav.)}|} = 
		\frac{ \left (  \frac{ k q_e q_p}{d^2} \right )}{\left (  \frac{ G m_e m_p}{d^2} \right )}
	\end{align*}

	Assumindo

	\begin{align*}
		\frac{|(\text{F. eletro.})|}{|\text{(F. grav.)}|} = 
		\frac{ \left (  \frac{ k q_e q_p}{d^2} \right )}{\left (  \frac{ G m_e m_p}{d^2} \right )}
	\end{align*}

	\chapter{Materiais Magnéticos}

	\section{Q7}

	Em algebra geométrica, isso é trivial. Queremos achar uma 1-forma $\xi$
	que sirva de potencial onde $\mathbf{H} = -\grad \xi$ onde
	$\Delta \xi = \grad \cdot \mathbf{M} = -\rho_m$.

	\vspace{\baselineskip}

	Sabendo que $\mathbf{H}$ é






\end{document}