\documentclass[a4paper]{book}


\usepackage{mlmodern}			% Usa a fonte Latin Modern			
\usepackage[T1]{fontenc}		% seleção de códigos de fonte.
\usepackage[utf8]{inputenc}		% determina a codificação utiizada (conversão automática dos acentos)
\usepackage[brazil]{babel}
\usepackage{hyperref}  			% controla a formação do índice
%\usepackage{parskip}			% espaçamento entre os parágrafos
\usepackage{nomencl} 			% Lista de simbolos
\usepackage{microtype} 			% para melhorias de justificação
\usepackage{booktabs}			% para tabelas
\setlength{\belowcaptionskip}{6pt} % espaçamento depois do título das tabelas
%\setlength{\abovecaptionskip}{6pt}

\hypersetup{
		colorlinks=true,       		% false: boxed links; true: colored links
		linkcolor=blue,        		% color of internal links
		citecolor=blue,        		% color of links to bibliography
		filecolor=magenta,     		% color of file links
		urlcolor=blue}

\IfFileExists{html.sty}
{\usepackage{html}}
{\usepackage{comment}
 \excludecomment{htmlonly}
 \excludecomment{rawhtml}
 \includecomment{latexonly}
 \newcommand{\html}[1]{}
 \newcommand{\latex}[1]{##1}
 \ifx\undefined\hyperref
  \ifx\pdfoutput\undefined \let\pdfunknown\relax
   \let\htmlATnew=\newcommand
  \else
   \ifx\pdfoutput\relax \let\pdfunknown\relax
    \usepackage{hyperref}\let\htmlATnew=\renewcommand
   \else
    \usepackage{hyperref}\let\htmlATnew=\newcommand
 \fi
  \fi
 \else
  \let\htmlATnew=\renewcommand
 \fi
 \ifx\pdfunknown\relax
  \htmlATnew{\htmladdnormallink}[2]{##1}
 \else
  \def\htmladdnormallink##1##2{\href{##2}{##1}}
 \fi
 \long\def\latexhtml##1##2{##1}}

% latex2html nao suporta \ifthenelse...
\usepackage[num,abnt-verbatim-entry=yes]{abntex2cite}
%overcite
%\citebrackets{}

\newcommand{\OKs}{173}
\newcommand{\quaseOKs}{12}
\newcommand{\nadaOK}{3}

\def\Versao$#1 #2${#2}
\def\Data$#1 #2 #3${#2}
%\newcommand{\bibtextitlecommand}[2]{``#2''}

\usepackage{color}
\definecolor{thered}{rgb}{0.65,0.04,0.07}
\definecolor{thegreen}{rgb}{0.06,0.44,0.08}
\definecolor{thegrey}{gray}{0.5}
\definecolor{theshade}{rgb}{1,1,0.97}
\definecolor{theframe}{gray}{0.6}

\IfFileExists{listings.sty}{
  \usepackage{listings}
\lstset{%
	language=[LaTeX]TeX,
	columns=flexible,
	basicstyle=\ttfamily\small,
	backgroundcolor=\color{theshade},
	frame=single,
	tabsize=2,
	rulecolor=\color{theframe},
	title=\lstname,
	escapeinside={\%*}{*)},
	breaklines=true,
	commentstyle=\color{thegrey},
	keywords=[0]{\fichacatalografica,\errata,\folhadeaprovacao,\dedicatoria,\agradecimentos,\epigrafe,\resumo,\siglas,\simbolos,\citacao,\alineas,\subalineas,\incisos},
	keywordstyle=[0]\color{thered},
	keywords=[1]{},
	keywordstyle=[1]\color{thegreen},
	breakatwhitespace=true,
	alsoother={0123456789_},
	inputencoding=utf8,
	extendedchars=true,
	literate={á}{{\'a}}1 {ã}{{\~a}}1 {é}{{\'e}}1 {è}{{\`{e}}}1 {ê}{{\^{e}}}1 {ë}{{\¨{e}}}1 {É}{{\'{E}}}1 {Ê}{{\^{E}}}1 {û}{{\^{u}}}1 {ú}{{\'{u}}}1 {â}{{\^{a}}}1 {à}{{\`{a}}}1 {á}{{\'{a}}}1 {ã}{{\~{a}}}1 {Á}{{\'{A}}}1 {Â}{{\^{A}}}1 {Ã}{{\~{A}}}1 {ç}{{\c{c}}}1 {Ç}{{\c{C}}}1 {õ}{{\~{o}}}1 {ó}{{\'{o}}}1 {ô}{{\^{o}}}1 {Õ}{{\~{O}}}1 {Ó}{{\'{O}}}1 {Ô}{{\^{O}}}1 {î}{{\^{i}}}1 {Î}{{\^{I}}}1 {í}{{\'{i}}}1 {Í}{{\~{Í}}}1,
}
\let\verbatim\relax
 	\lstnewenvironment{verbatim}[1][]{\lstset{##1}}{}
}

\usepackage[T1]{fontenc}		
\usepackage[utf8]{inputenc}		
\usepackage{indentfirst}	

\usepackage{listings}
\usepackage[table,xcdraw]{xcolor}			
\usepackage{graphicx}	
\usepackage{physics}

\usepackage{microtype}
\usepackage{amsfonts}
\usepackage{mathtools}
\usepackage[makeroom]{cancel}

\usepackage{siunitx}
\usepackage{amsthm}
\usepackage{amsmath}
\usepackage{amssymb}

\usepackage{nicematrix} %Matrizes bonitas
\usepackage{xkcdcolors} %Cores do xkcd
\usepackage{tikz, tcolorbox, bclogo} %Desenhos e afins
\usepackage{empheq} %Caixas ao redor de equações em um align

\usepackage{float}
\usepackage{caption}
\usepackage{cleveref}
\usepackage{epigraph} 

\usepackage{tabularx}
\usepackage{array}
\usepackage{hyperref}
\usepackage[Sonny]{fncychap}


\numberwithin{equation}{chapter}

\definecolor{blue}{RGB}{41,5,195}

\frenchspacing

\begin{document}

\newcommand{\titulo}{\textbf{Ian Bell's notes}\\ \textbf{Ii : Multivector Calculus}}
\newcommand{\abnTeX}{abn\TeX}

%begin{latexonly}
\title{\titulo}

\author{
Artur R. B. Boyago
}
\date{\today}

\maketitle

\tableofcontents

\section{Preface}

All of the content here originally comes from \href{http://www.iancgbell.clara.net/maths/geoalg.htm}{Ian Bell's
stunning 1999 page}. However,
due to the outdated graphic design and possible depredation of the site in question, there is a 
risk of the material eventually becoming unreadable, thus the PDF transcription.

\vspace{\baselineskip}

Further, I'd like to add lots of personal notes on this kind of mathematics from  studies
in the matter in order to enhance the original material, so there will be undisclosed
differences, as well as interjections in \textcolor{gray}{gray color}.

The book is formatted in the same order as the original
with all content reproduced in modern \LaTeX format. Many thanks, Artur.




\chapter{Multivector Derivatives}
\epigraph{Calculus required continuity, and continuity was supposed to require the infinitely little; but nobody could discover what the infinitely little might be.}{\textit{Bertrand Russell}}

\section{Introduction}

Multivector calculus is of interest for modelling fluidic flow, gravitational fields, and so
forth. The intention here is to provide a quick inroad into the subject rather than a full 
and formal presentation. For a rigourous approach, see Hestenes and Sobcyk.

\vspace{\baselineskip}

This document assumes familiarity with Multivectors. Notations defined in that document are 
retained here. Note that we here use labels $\mathbf{e_1,e_2,\cdots e_n}$
to denote a typically fixed, "base", "universal", "fiducial" frame 
and $\mathbf{h}_{i_\mathbf{p}}$ to denote tangent vectors. In much of the literature,
$\mathbf{e_i}$represent tangent or otherwise "motile" vectors while 
$\sigma_i$ or $\gamma_i$ represent a "base frame".
This document makes extensive use of subscripts and superscripts to indicate dependencies 
usually "dropped" in conventional treatments and is, in consequnce, theoretically ambiguous. 
Does $\mathbf{v}_{i\mathbf{p}}$ , for example, mean that $\mathbf{v}_i$ 
is defined over or dependant on $\mathbf{p}$, 
or that $\mathbf{v}$ is a function of $i_{\mathbf{p}}$? 
In practice, meanings will be clear in context.

\vspace{\baselineskip}

Tensors are traditionally a difficult concept but multivectors make them far easier to
understand, manipulate, and generalise. They are fundamental to many applications so we
address them here.


\section{Multivector functions as tensors}

The traditional presentation of an $n$-dimensional tensor of integer rank $r$ is 
a point-dependant $n^r$ element "array" or "matrix" defined with respect to a given 
$n$-dimensional coordinate frame, that transforms according to particular rules in 
accordance with transforms of the underlying coordinate frame. 

\vspace{\baselineskip}

Multivectors provide an attractive alternative (and more general) formulation under which the
conventional tensor product follows directly from the geometric product. More formal 
definitions of the following explicitly specify a "scalar source" from which to "build" linear combinations, but here we implicitly assume "real" scalars (from $\mathbb{R}$ or a 
(finite-precision) approximation thereof). 

\subsection{Fields}

A \textbf{field} is a function $\mathbf{f} : \mathbb{R}^{p,q,r} \rightarrow 
\mathcal{R}_{p,q,r}$, that is, it maps one point from the $\mathbb{R}^{p,q,r}$
vector space and returns a multivector. 
In other words: a \emph{point-dependant multivector}. If 
the function is (unit) $k$-vector valued, we have a (unit) $k$-field. 
A 0-field thus associates a scalar value with every point \textcolor{gray}{(Say temperature)},
and a 1-field associates a 1-vector with every point \textcolor{gray}{(The electric field)},
and a 2-field associates a 2-vector with every point \textcolor{gray}{(The magnetic field)}.

\vspace{\baselineskip}

From a programmers' perspective, fields are functions having at least one 1-vector parameter.
This "primary" parameter is usually interpreted as a point or position. When the primary 
argument is interpreted as a (scaled) direction rather than a point we will refer here to a 
\emph{directional} $k$-field. 

\subsection{Tensors}


\begin{tcolorbox}[colback=white, colframe=blue!10!black, title=\textbf{Tensor definition} ]

We regard an $n$-dimensional tensor of degree $k$ as a point-dependant multilinear 
$n$-dimensional multivector-valued function of $k$ $n$-dimensional 1-vectors. 

\begin{align*}
    \mathbf{f_x} : (\mathbb{R}^{p,q,r})^k \rightarrow \mathcal{R}_{p,q,r}, \quad p+q+r=n
\end{align*}

\end{tcolorbox} 

\vspace{\baselineskip}

By \emph{multilinear} we mean linear in each argument. If $k=0$ \textcolor{gray}{($\mathbb{R}^0 \simeq \mathbb{R}$)}
we have a point-dependant function taking no arguments and returning a (point-dependant) multivector, effectively a field.
A $(t,0)-$tensor is thus a $t-$field, often refered to as an invariant tensor, though its "value" does in general vary with 
$\mathbf{x}$.

\vspace{\baselineskip}

If $ \mathbf{f_x}\mathbf{(a_1, a_2, \cdots a_k)} =
\langle \mathbf{f_x}\mathbf{(a_1, a_2, \cdots a_k)} \rangle_t $
(ie. $\mathbf{f_x}$  is $t-$vector valued)
we say the tensor has type $t$ and rank $(t+k)$ 
and refer to it here as a $(t,k)-$tensor.

\vspace{\baselineskip}

\textcolor{gray}{
In conventional tensor calculus, the order $(n,m)$ is given as the 
number of vectors and forms (elements of the dual space) 
$(V\oplus V \cdots V^* \oplus V^*)$ that compose the tensor, and the return type.
In this case, a $(0,0)-$tensor returns a 0-vector and has a rank of $0$, and
$(1,0)-$tensor returns a common 1-vector and has a rank of 1, and so forth. Now,
often we may think of a $(0,k)$ tensor then as returning a $k-$form rather than a
$k-$blade. That is, rather than an oriented volume we get the classic oriented
hyperplane set that act as linear functionals of the dual space $V^*$.}

\vspace{\baselineskip}

\textcolor{gray}{For the sake of a full ´multivector´ interpretation, we may, however,
interpret $(0,k)-$tensors as, in 3D space for example, a collection of $k$ three dimensional
1-vectors that, in this case, may function as the analogue linear function by means of a reciprocal
basis. \href{https://math.stackexchange.com/questions/3447049/intuitive-understanding-of-2-forms-1-1-tensors-and-other-fundamental-objects}{This mathstackexchange question also clarifies}
certain doubts you may have.}

\vspace{\baselineskip}

From a programmers' perspective, tensors are multivector-valued functions of 
at least one 1-vector argument, linear ("affine") in all but the primary argument.
When $t=k$ we refer to a $k-$tensor rather than a $(k,k)-$tensor.
A $k-$tensor is thus a point-dependant $k-$vector-valued multilinear 
function of $k$ 1-vectors. 
In particular, a 1-tensor is a point-dependant directional 1-field.

\textcolor{gray}{Now, here we may see mixed degree tensors such as $(1,1)$
as outputting a single 1-vector, but taking differential 1-forms as inputs as well.
The total degree of this tensor is 2, and so we see we'd need \emph{two} indices
to operate on other vectors, so something like $A_i^j$. where the $i$ index
is below representing the 1-vector (covariant) component, and the $j$ index is
representing the 1-form (contravariant) component. So a $(2,1)$ tensor, which has
degree 3, would require $A_{ij}^l$, where both $i,j$ would act for our bivector
output and the $l$ would serve for our 1-form.}

\vspace{\baselineskip}

The "scalar product" $\lrcorner$ is a $(0,2)-$tensor, though we usually write 
$\mathbf{a\lrcorner b}$ in preference to $\lrcorner(\mathbf{a,b})$ \textcolor{gray}{(or $\mathbf{g(a,b)}$)}. 
The outer product $\wedge$ is a 2-tensor. 
The geometric product is a tensor of degree 2 but "mixed" type. 


\subsection{Forms}

If a $(t,k)-$tensor is \emph{skewsymmetric} in its arguments so that:

\begin{align*}
    \mathbf{f_x(a_1,a_2,\cdots a_k} = \mathbf{L_x(a_1 \wedge a_2 \wedge \cdots a_k)} = \mathbf{L_x(A_k)}
\end{align*}

Can be viewed as a function of a single $k-$blade rather than of $k$ 1-vectors,
then it is called a skewsymmetric $(t,k)$-tensor or a $(t,k)$-multiform. 
When $t=k$ we abbreviate to a $k-$multiform. If $t=0$ 
(ie. $\mathbf{f_x}$  is scalar valued) then it is instead called a \textbf{$k$-form.} 
A 1-multiform is a 1-tensor. It can be shown [see Hestenes & Sobczyk] that any $k-$form
can be expressed as:

\begin{align*}
    L_{\mathbf{x}}(\mathbf{A}_k) = \mathbf{u_k \lrcorner A_k}
\end{align*}

Where  $\mathbf{u_k}$ is a point-dependant $k-$vector.

\vspace{\baselineskip}

If a $k$-multiform maps any given $k$-blade to another $k$-blade 
(rather than to a $k-$vector) then we say the multiform is \emph{blade preserving.}
A 1-tensor is thus a blade-preserving 1-form since any 1-vector is a 1-blade. 
It can be shown that provided $k \ne n/2$, any blade preserving $k-$form
is merely the outermorphism of a 1-form. For $k=n/2$, the geometric 
dual prserves $k$-blades but is not an outermorphism.


\subsection{Dyads}

A $k-$dyad is a $k-$multiform of the form:

\begin{align*}
    D(\mathbf{A}_k) = \mathbf{u}_k ( \mathbf{v}_k \lrcorner \mathbf{A}_k)
\end{align*}

Where $\mathbf{u}_k, \mathbf{v}_k$ are point-dependant $k-$blades. 
A $k-$multiform can be expressed in dyadic form as a sum of $k-$dyads. 
A 1-dyad is known as a dyad. A 0-dyad is the "succesive" multiplicative 
combination of two scalar fields $D_{\mathbf{x}}(a)=u_{\mathbf{x}}v_{\mathbf{x}} a$.

\subsection{Multitensors}

We can generalise a (t;k)-tensor to a (t;k)-multitensor) being a point dependant multivector-valued function of k multivectors Fp(a1,a2,...,ak) which is t-vector-valued when acting on k 1-vectors. , linear in all but the primary (point) argument.

\vspace{\baselineskip}

We will henceforth use the term tensor to refer to a multilinear multivector-valued function of k nonprimary 1-vector arguments and multitensor for a multilinear multivector-valued function of k nonprimary multivector arguments.

\vspace{\baselineskip}

 We will typically restrict the grade of the nonlinear "primary" multivector argument p to 1 and consider it as a 1-vector "point" p . If a multitensor is t-vector valued, we can regard it as a sum of (t;k)-forms with k ranging from 0 to N. 

\subsection{Extended Fields}
\subsection{Outermorphisms and Determinants}
\subsection{Eigenblades}
\subsection{Coordinate-based Tensor Representations}




\section{Differentials}
There are a number of different multivector derivatives and they can be approached in a 
variety of ways. When defining any particular directed derivative ( Ða say) it suffices to
define its action on scalars and 1-vectors since the product rule 
$\grad $ then provides the directed differentiation of general multivectors. 


\subsection{Directed Scalar Derivatives}
\subsection{Linearity of the Differential}
\subsection{Differentiating Exponentials}
\subsection{The Directed Chain Rules}
\subsection{Primary Differential}
\subsection{Second Primary Differential}
\subsection{Third Primary Differential}
\subsection{Secondary Differential}
\subsection{Lie Product}


\section{Undirected derivatives}
\section{Multivector Fractals}



\chapter{Multivector Manifolds}
\section{Curves and Manifolds}
\section{Integration over a $M-$Curve}
\section{Fourier Transform}
\section{Differentiation within a $M-$Curve}
\section{Fundamental Theorem of Calculus}
\section{Poles and Residues}

\chapter{Manifold Restricted Tensors}
\section{Further differentials and derivatives}
\section{Curvature 2-tensor}





\chapter{The Coordinate Based Approach}

\section{Streamline coordinates}
\subsection{Introduction}

\section{Lie Derivatives}
\subsection{Lie Bracket}
\subsection{Lie Drag}
\subsection{Lie Derivative}
\subsection{Covariant Frame}
\subsection{Covectors}
\subsection{Parallel Transport}
\subsection{Linear Connection}
\subsection{Directed coderivative}
\subsection{Geodesics}

\section{Curvature}
\subsection{1-Curvature}
\subsection{2-Curvature as ortho bi-circle}
\subsection{Curvature Measures}
\subsection{Curvature Measures of Polyhedral 2-curves}
\subsection{Curvature as loop integrative limit / Holonomy}
\subsection{Curvature as Coderivative Operator}
\subsection{Symmetries and Bianchi Relations}

\section{Torsion}







\bibliographystyle{abbrv}
\bibliography{referencias}

    
    

\end{document}